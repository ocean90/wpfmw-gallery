
\documentclass[a4paper,12pt,oneside]{article} %einseitiger Druck
% Optionen:
% - a4paper => DIN A4-Format
% - 12pt    => Schriftgröße (weitere  
%         grundlegende Fontgrößen: 10pt, 11pt)
% - oneside => Einseitiger Druck

%% Verwendete Pakete:
\usepackage[ngerman]{babel} % für die deutsche Sprache
\usepackage{caption} % Für die Titel in Gleitobjekten (wie figure)
\usepackage[T1]{fontenc} % Für Sonderzeichen u.a.
\usepackage[utf8]{inputenc} % Für die direkte Eingabe von Umlauten
\usepackage{fancyhdr} % Für Kopf- und Fußzeilen
\usepackage{lscape} % Für Querformat
\usepackage{lmodern} % Type1-Schriftart für nicht-englische Texte
\usepackage{ulem} % Für das Unterstreichen von Text z.B. mit \uline{}
\usepackage[left=3cm,right=2cm,top=1.5cm,bottom=1cm,textheight=245mm,textwidth=160mm,includeheadfoot,headsep=1cm,
			footskip=1cm]{geometry} % Einrichtung der Seite 

\usepackage{graphicx} % Zum Laden von Graphiken

% Pakete für Tabellen
\usepackage{tabularx} % Einfache Tabellen
\usepackage{longtable} % Tabellen als Gleitobjekte (für die Aufteilung bei langen 
 %Tabellen über mehrere Seiten)
\usepackage{multirow} % Für das Verbinden von Zeilen innerhalb einer Tabelle mit
 % \multirow{anzahl}{*}{Text}

% Pakete für Formeln
\usepackage{amsmath}
\usepackage{amsthm}
\usepackage{listings}
\usepackage{amsfonts}
\usepackage{setspace} % Paket zum Setzen des Zeilenabstandes
\usepackage[colorlinks,pdfpagelabels,pdfstartview=FitH,
bookmarksopen=true,bookmarksnumbered=true,linkcolor=black,
plainpages=false,hypertexnames=false,citecolor=black]{hyperref} 
% Globale Einstellungen (gelten für das gesamte Dokument, sofern sie nicht verändert werden)
\setlength{\headheight}{14.599pt} % Kopzeilenhöhe festlegen
\setlength{\headwidth}{160mm}     % Kopzeilenbreite festlegen
\onehalfspacing % Zeilenabstand auf 1,5-zeilig setzen

 
%%%%%%%%%%%%%%%%%%%%%%%%%%%%%%%%%%%%%%%%%%%%%
%% DOKUMENT                                %%
%%%%%%%%%%%%%%%%%%%%%%%%%%%%%%%%%%%%%%%%%%%%%
\begin{document}
 
  % TO-DO: Unbeschriftetes Vorblatt (Leere Seite)
  % ...
   
  % Deckblatt 
  \pagestyle{empty}
  \section*{} %Sternchen damit die Seite nicht im Ihnhaltsverzeichnis auftaucht.
  
  \begin{titlepage}
    \includegraphics[scale=1.00]{Sources/logo_fhkoeln_sw}\\
    \begin{center}
      \Large
      Fachhochschule Köln\\
      Fakultät für Informatik und Ingenieurwissenschaften\\
      \hrule\par\rule{0pt}{2cm} % Horizontale Trennlinie  mit 2 cm Abtand nach unten erzeugen
      \LARGE
      \textsc{Documentation}\\
      \vspace{1cm} % Vertikaler Abstand von 1cm erzeugen
      \huge
      WPF Modern Web\\
      \Large
      \vspace{1.5cm}
      \large
      presented at Cologne University of Applied Science\\
      Campus Gummersbach\\
      Course of studies
      Medieninformatik\\ 
      \vspace{1.0cm}
      prepared by:\\
      \textsc{Dominik Schilling}\\
      (Matrikelnummer: ???)\\
      \textsc{Laura-Maria Hermann}\\
      (Matrikelnummer: 11083968)\\
      \textsc{Dario }\\
      (Matrikelnummer: ???)\\
      \vspace{1.5cm}
      \vspace{1.5cm}
     
    \end{center}    
  \end{titlepage}
  
  \newpage
 
  
  \tableofcontents
  % TO-DO: Seitennummerierung "einschalten"
  % ...
	\pagestyle{fancy}
   		
 \newpage %neue Seite zwischen Inhaltsverzeichnis und Abbildungsverzeichnis
  % TO-DO: Abbildungsverzeichnis einfügen
  % ...
  
 

  

\newpage

\section{Introduction}

In this documentation are the elaborated content of the WPF modern
Web application. Is documented, among others, the source code but
also approaches or considerations which have been made in advance.
The task was to create a gallery in the form of a website, which had
to include the following:
\begin{itemize}
\item The gallery as such should be implemented using Javascript. 
\item Individual images should be zoomed by clicking and navigable by arrows. 
\item There should be a log-in available area. 
\item MySQL connection to a LogIn for the others but also to add new images 
\item JSON, JQuery and AJAX should be used.
\end{itemize}

\section{Planning}

The first task was to think exactly how the gallery will look like
- not optically but functional. Because a log-in area was prescribed
it seemed sensible to realize an administrator who got an extended
functionality (which is initially irrelevant) 

When considering existing galleries on the net you could roughly distinguish
between two groups. 

On one hand, web pages which serve as a real gallery and only have
one author who may publish works. This would be only an admin, the
author himself who upload photos, delete, ... .

On the other hand, many galleries will be held as a kind of platform.
These are in addition to the administrator other users who register
very simple and thus can upload their photos.

As part of the WPF's it was decided to create the second variant. 


\section{Realization}


\subsection{MVC-paradigm}

First is to say, that was held during the entire implementation of
the MVC paradigm. This is explained in more detail below and explain
its components. 

\emph{What is MVC? }

MVC (model, view, controller) is an architectural structure of software,
which provides a strict division of tasks and separation. A distinction
is made between the three components model, view and controller. These
are explained in more detail below.


\subsubsection*{Request}

Example: 
\begin{lstlisting}
GET localhost/user/?foo=bar
\end{lstlisting}

 The Request class extracts the URL into its component parts. user,
for example, is one segment.


\subsubsection*{Controller}

The controllers include the logic always belongs to a resource. 

Example: 

\begin{lstlisting}
localhost / user
\end{lstlisting}
then the controller is loaded, see load.php. The controllers are
located in "includes / controllers" directory. If no other resource
specify, for example "localhost / user / blaa"
 is called the index method of the controller will start. If the resource was
"blaa" and indicate the method exists, it is called when it does not
exist, the "index" is retrieved.


\subsubsection*{Views}

The views are located in the 
includes / views
 directory. The views are responsible for the visual representation
of data.

Example call:

\begin{lstlisting}
$view = new View( 'install/success' ); 
$view->set_page_title( 'Installation' );  
$view->render(); // Gibt das Template aus
\end{lstlisting}

A view can consist of several template, eg headers - content - footer.


\subsubsection*{Model}


\subsection{Database }

One of the first tasks is creating a database in which all uploaded
images are stored and can be retrieved. In the following, all the
contents and information related to the database are examined in detail.


\subsubsection{Create}

First, to create a database.  Each user can later use the graphical
interface to create a gallery and thus create a set of tables in the
database. Following you are able to find explanations of all realized
tables and fields. Collation: utf8\_general\_ci


\subsubsection{Database tables}


\paragraph*{Users}

This is to inform the users of the gallery. These are uniquely identified
by an ID.
\begin{tabular}{|c|c|}
\hline 
Feld & Typ\tabularnewline
\hline 
\hline 
ID & bigint(20)\tabularnewline
\hline 
user\_login & varchar(50)\tabularnewline
\hline 
user\_pass & varchar(64)\tabularnewline
\hline 
user\_email & varchar(100)\tabularnewline
\hline 
user\_registered & datetime\tabularnewline
\hline 
user\_status & int(10)\tabularnewline
\hline 
\end{tabular}


\paragraph*{Usermeta}

Metadatas of the users

\begin{tabular}{|c|c|}
\hline 
Feld & Typ\tabularnewline
\hline 
\hline 
ID & bigint(20)\tabularnewline
\hline 
user\_id & bigint(20)\tabularnewline
\hline 
meta\_key & varchar(255)\tabularnewline
\hline 
meta\_value & longtext\tabularnewline
\hline 
\end{tabular}

\paragraph*{Images}

Data of the images
\begin{tabular}{|c|c|}
\hline 
Feld & Typ\tabularnewline
\hline 
\hline 
ID & bigint(20)\tabularnewline
\hline 
user\_id & bigint(20)\tabularnewline
\hline 
uploaded\_date & datetime\tabularnewline
\hline 
image\_caption & longtext\tabularnewline
\hline 
image\_title & text\tabularnewline
\end{tabular}

\paragraph*{Imagemeta}

Metadatas of the images

\begin{tabular}{|c|c|}
\hline 
Feld & Typ\tabularnewline
\hline 
\hline 
ID & bigint(20)\tabularnewline
\hline 
image\_id & bigint(20)\tabularnewline
\hline 
metakey & varchar(255)\tabularnewline
\hline 
meta\_value & longtext\tabularnewline
\end{tabular}

\paragraph*{Gallery}

Informations about the gallery in general

\begin{tabular}{|c|c|}
\hline 
Feld & Typ\tabularnewline
\hline 
\hline 
ID & bigint(20)\tabularnewline
\hline 
user\_id & bigint(20)\tabularnewline
\hline 
public & boolean\tabularnewline
\hline 
gallery\_title & text\tabularnewline
\hline 
gallery\_description & longtext\tabularnewline
\end{tabular}

\paragraph*{ImageRelationships}

\begin{tabular}{|c|c|}
\hline 
Feld & Typ\tabularnewline
\hline 
\hline 
ID & bigint(20)\tabularnewline
\hline 
gallery\_id & bigint(20)\tabularnewline
\end{tabular}

\subsubsection{SQL-Query}

The following SQL commands are called to create the tables used commands.

\begin{lstlisting}
CREATE TABLE `gallery` (
  `ID` bigint(20) unsigned NOT NULL,
  `user_id` bigint(20) unsigned NOT NULL,
  `is_public` tinyint(1) NOT NULL DEFAULT '1',
  `gallery_title` text NOT NULL,
  `gallery_description` longtext NOT NULL,
  PRIMARY KEY (`ID`)
) DEFAULT CHARSET=utf8;

CREATE TABLE `imagemeta` (
  `ID` bigint(20) unsigned NOT NULL AUTO_INCREMENT,
  `image_id` bigint(20) unsigned NOT NULL,
  `meta_key` varchar(255) NOT NULL,
  `meta_value` longtext NOT NULL,
  PRIMARY KEY (`ID`),
  KEY (`image_id`),
  KEY (`meta_key`)
) DEFAULT CHARSET=utf8;

CREATE TABLE `images` (
  `ID` bigint(20) unsigned NOT NULL AUTO_INCREMENT,
  `user_id` bigint(20) unsigned NOT NULL,
  `uploaded_date` datetime NOT NULL,
  `image_title` text NOT NULL,
  `image_description` longtext,
  `image_caption` text NOT NULL,
  PRIMARY KEY (`ID`),
  KEY (`user_id`)
) DEFAULT CHARSET=utf8;

CREATE TABLE `image_relationships` (
  `image_id` bigint(20) unsigned NOT NULL,
  `gallery_id` bigint(20) unsigned NOT NULL,
  PRIMARY KEY (`image_id`, `gallery_id`),
  KEY (`gallery_id`)
) DEFAULT CHARSET=utf8;

CREATE TABLE `usermeta` (
  `ID` bigint(20) unsigned NOT NULL AUTO_INCREMENT,
  `user_id` bigint(20) unsigned NOT NULL,
  `meta_key` varchar(255) NOT NULL,
  `meta_value` longtext NOT NULL,
  PRIMARY KEY (`ID`),
  KEY (`user_id`),
  KEY (`meta_key`)
) DEFAULT CHARSET=utf8;

CREATE TABLE `users` (
  `ID` bigint(20) unsigned NOT NULL AUTO_INCREMENT,
  `user_login` varchar(60) NOT NULL,
  `user_pass` varchar(64) NOT NULL,
  `user_email` varchar(100) NOT NULL,
  `user_registered` datetime NOT NULL DEFAULT '0000-00-00 00:00:00',
  `user_status` int(10) NOT NULL,
  PRIMARY KEY (`ID`),
  KEY (`user_login`),
  KEY (`user_email`)
) DEFAULT CHARSET=utf8;
\end{lstlisting}





\subsection{resources }

There follows an explanation of the necessary resources.


\subsubsection{/register}

At this point, the registration of new users is possible. This only
needs to be done so far when create a gallery and want to post pictures.
Registration information such as name, password and email address
are required.


\subsubsection{/login}

The direct login area if you are already a registered user.
\begin{itemize}
\item Login
\end{itemize}

\subsubsection{/logout}

The direct logout area if you have previously logged in.
\begin{itemize}
\item Logout
\end{itemize}

\subsubsection{/profile}

His own profile. At this point the user gets to make any profile settings
and changes. This would be for example loading up an avatar.
\begin{itemize}
\item Setting
\item Avatar
\item Mail
\item Name
\item ...
\end{itemize}

\subsubsection{/user/\{\$id\}}

With the help of this resource you are able to get the profile of
a specific user (which is uniquely identified by its ID). This is
especially useful for viewing all of the galleries of the user.
\begin{itemize}
\item User profile 
\item Overview of public / private galleries
\end{itemize}

\subsubsection{/user/\{\$id\}/galleries/\{\$id\}}

If you want to directly get a concrete gallery of a user, you have
to specify the ID of the gallery as a query parameter.
\begin{itemize}
\item Gallery of a single user
\end{itemize}

\subsubsection{/search}

Use the search function, you can get special content. These are selected
with filters.
\begin{itemize}
\item Search
\item Filters
\end{itemize}

\subsubsection{/install}

At the beginning of a session, the installation must be carried out.
\begin{itemize}
\item Installation routine
\end{itemize}

\subsection{Code}

In the following, the program code is explained and illustrated.


\subsubsection{Controllers}

How to recognize some control classes were implemented and used. 

Almost all of these controller classes differ only in their title.
The page title, which is visible in the browser connection is \textquotedbl{}set\textquotedbl{}
at this point. 

For example, the title of 'class-error controller' on \textquotedbl{}404
- Page does not exists!\textquotedbl{} placed. Another was described
in controller.


\subsubsection{Form-Validation}

Especially in all forms, such as the Login or Register, it was important
to make some measures for validation.

Especially in all forms, such as the Login or Register, it was important
to take some measures for validation. When logging in to be examined,
first, whether the username is available and secondly whether the
associated password is correct.

When registering on the one hand to make it, that the username and
email are not forgiven, and on the other hand, ensure that both passwords
entered are the same.

These errors treatments are implemented in the class \textquotedbl{}class-user-manager\textquotedbl{}
and shown by the functions ,,validate\_new\_user``, ,,set\_current\_user``
and ,,create\_user through``.

First, a variable is created for each field of the form. If this is no longer empty after entering of the user, the value is what the user has entered fetched and processed using the POST method. In addition, it is converted by the htmlspecialchars function. This is particularly important in order to maintain security. In general this function transform, with use of the selected flags,  single characters, so that they are not interpreted as false. In this context, it is particularly the quotation marks, which are converted to  to prevent misunderstandings as <input value="Mein" Name"> (the value would only be "Mein" and Name" would be considererd as invalid HTML-Object).


\subsubsection{CSS-Framework Bootstrap}

As a basis and simplify the work was resorted to the free bootstrap
framework. 

This simplified, especially at the beginning, the realization and
getting huge. Bootstrap has the advantage to be platform independent,
among others but also support in its latest version of the \textquotedbl{}responsive
design\textquotedbl{}. The site got a dynamic structure, regardless
of which device you use.

\subsubsection{Functions}
In the following there are declarations about the used functions writen in the functions.php.
In addition it should be said here that only roughly functionality is explained. Further descriptions can be found in the source code.

At first there was a function for checking the compatibility of the php-Version. Its important to tell the user, if it is like this, that he has an older php-Version or not started the MySQLi extension. 

Another functions checks if the installation was already complete. If this isn't the case it redirects to the installation page. 

To guarantee the safety of the passwords was implemented a hash function. 
For this purpose, the existing \hyperlink{http://www.openwall.com/phpass/}{hash function framework}  was used. 

\subsubsection{Javascript}
It follows a short description of the javascript functions which was implemented. 

\hyperlink{https://github.com/ocean90/wpfmw-gallery/blob/master/assets/js/password-strength.js}{Password strength function}
As an extra to the user a feature was implemented, which determines the strength of the password
\end{document}

